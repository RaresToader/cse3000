\begin{abstract}
Distributed reflection denial-of-service (DRDoS) attacks are a type of cyberattack where a malicious actor sends requests to public and open servers on behalf of the victim by spoofing their IP address. The traffic generated by the corresponding responses is directed towards the victim (whose IP address appeared as the source address of the initial packets), potentially exhausting their bandwidth. These attacks have kept becoming more powerful over the years. 

This thesis presents a measurement study of three well-known protocols, where we assess the amplification potential of hosts located in Greece running these protocols. We find that DNS remains the most vulnerable protocol to amplification; the top 250 hosts can cumulatively amplify the traffic by 32,000$\times$. Furthermore, we discover that the ``ANY'' query type and the improperly configured DNS extension (EDNS0) are two significant causes of DNS amplification. Lastly, we also find hosts vulnerable to looping attacks, a novel threat in the context of DDoS attacks. 

% The aim of this template is to make it more clear what is expected from you. 
% \textbf{It is by no means required to follow this exact same structure.}
% The abstract should be short and give the overall idea:
% what is the background, the research questions, what are your contributions, and what are the main conclusions.
% It should be readable as a stand-alone text (preferably no references to the paper or to outside literature).
\end{abstract}