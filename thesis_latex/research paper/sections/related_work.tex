\section{Related Work}

\looseness=-1 This section describes existing research in the field of amplification attacks. Rossow~\cite{amplification_hell} did a comprehensive study of how suitable 14 popular UDP-based protocols are for amplification attacks. The results of this study were quite worrying, as the author found millions of amplifiers that could be used in a distributed attack for six protocols. We similarly assess the amplification potential of DNS and NTP. The paper also explores methods to detect DRDoS attacks in the wild using traffic analysis. Lastly, Rossow also presents  DRDoS mitigation techniques and ways to strengthen network protocols against this type of attack. 


\looseness=-1 Griffioen et al.~\cite{griffioen_scan_2021} explored the steps an attacker takes when orchestrating an amplification attack by observing a set of honeypots deployed in the wild. We followed a methodology akin to the one the authors presented. Their work also discovered that attackers have a shared memory of previously used servers in amplification attacks. Still, attackers differ in how they orchestrate such an attack, the landscape being filled by both novice actors and very sophisticated adversaries. 


\looseness=-1 Van der Toorn et al.~\cite{van_der_toorn_anyway_2021} researched the amplification potential of domain names in the context of ``ANY'' queries, and what the reduction in response size would be if ``ANY'' queries would be dropped, according to the suggestions of RFC 8482~\cite{rfc-8482}. They also proposed and validated a technique to approximate the size of responses to ``ANY'' requests based on a large active DNS measurement dataset. We crafted our DNS query (the one we measure the amplification factor with) following what the authors also proposed. 

\looseness=-1 Kührer et al.~\cite{exit_hell} studied the diversity of amplification sources, finding that they spread across various operating systems. A crucial impact of their paper was that they carried out a large-scale campaign to alert NTP administrators about the amplification potential of answering the ``monlist'' request. As a result, this campaign led to more than a 92\%  decrease in vulnerable NTP servers. Their work also highlights vulnerabilities in the TCP handshake that attackers could use as amplification vectors, which we will not explore as it is outside the scope of our research. They conclude that the root cause for amplification attacks is the fact that there are networks that allow IP address spoofing. 

\looseness=-1 Pan et al.~\cite{cispa-loopy} researched a novel and hazardous type of denial-of-service attack at the application layer. The \textit{looping attack} allows a malicious actor to send one IP-spoofed packet to a server that, in turn, triggers an infinite loop of back-and-forth messages between the server to whom the packet was sent and the server whose IP was spoofed in the initial request. This essentially enables infinite amplification. They propose a systematic methodology to search for vulnerable server pairs. They further use this methodology to investigate loops that can be formed with servers running various UDP protocols. The results were quite jarring, as an attacker could potentially abuse billions of server pairs for looping attacks. For this reason, we chose to study if hosts located in Greece are also susceptible to this type of attack. 



% \looseness=-1 Wagner et al.~\cite{wagner_united_2021} investigate the potential of enhanced mitigation through collaborative efforts at the Internet Exchange Points (IXPs) level. The central assumption they prove is that sharing data between IXPs leads to a higher percentage of detected DDoS attacks. Over 120k attacks are analysed, and a significant finding is that more than 80\% of these attacks remain undetected by local mechanisms due to imperfect local thresholds and the complex nature of attacks using multiple protocols. Their collaborative approach enhanced local attack detection, reducing up to 90\% more attack traffic. This highlights the role that core internet infrastructures (such as IXPs) play in detecting and mitigating DDoS attacks. We do not analyse this technique further. However, we underline its effectiveness and its positive effect on reducing the impact of DRDoS attacks.