\section{Discussion}

\looseness=-1 We have noticed that ``ANY'' and the EDNS0 buffer size of 4,096 are the main culprits of DNS amplification attacks. We have not found significant amplification potential for NTP and Memcached, likely due to effective patching campaigns held in the past~\cite{exit_hell}. This does not imply that these protocols are not dangerous around the world. Thus, we still recommend patching NTP to turn off private queries (which is the case for ``ntpd'' versions of 4.2.8 and above~\cite{ntpd-patched}) and for Memcached to disable the UDP port by default~\cite{memcached-fixed}, by upgrading the vulnerable version to version ``1.5.6'' or above~\cite{memcached-fixed-version}.

\looseness=-1 \textbf{Peer outcomes}. The influence of the EDNS0 buffer size on BAF in the context of DNS amplification spans other countries as well, namely Belgium (BE), Luxembourg (LU), France (FR), Sweden (SE) and the Netherlands (NL). However, the distribution of which values for the buffer size are advertised differs from country to country; for instance, hosts located in NL choose a size of 1,232 way more often than a value larger than 4k (almost $5\times$ more often), showing a better adoption of the measures suggested by~\cite{dnsflagday}. In our case, a value larger than 4k was advertised more than $6\times$ compared to 1,232. We also see 1,232 being advertised in the majority of cases in FR. The BAFs we found for DNS are similar to the ones observed in BE, LU and SE. Moreover, recursive NS are more easily weaponizable in these countries than authoritative NS. When analysing NTP and Memcached amplification, FR and NL are more vulnerable than any other country from our peers, with hosts that run these protocols and lead to high BAFs. For FR, we conclude that this might be the case due to the sheer number of hosts, the likelihood of finding exploitable hosts being thus larger. Lastly, BE, FR, and SE also showcase the Linux NTP servers, which leads to large BAFs via Mode 7 queries.


\looseness=-1 We found similar results to what others have discovered in previous work~\cite{van_rijswijk-deij_dnssec_2014}; they observed that the majority of hosts that adopted EDNS0 advertise a buffer size of at least 4,000 bytes, namely in 70\% of our hosts. We noted the tendency for the BAF to increase as the buffer size becomes larger. Another interesting finding was that we even received larger responses than the advertised buffer size, which should not be the norm. As RFC9210~\cite{rfc9210} describes, when the data does not fit within the buffer size, the responding server should set the truncated flag to true in its response, and the requesting server should retry over TCP. Thus, the size of the answer should not exceed the buffer size. We concluded that two DNS implementations, MikroTik DNS and Raiden DNSD, set a default buffer size value of 4,096. Hosts running these two also got the highest BAFs out of all our DNS hosts, making up 70\% out of the top 250 of our hosts. Reflecting on our DNS results, we want to highlight that BAFs between 120 and 130 are readily achievable by attackers since relevant sources cite smaller values for DNS amplification. For instance, RFC5358~\cite{rfc-5358} states that values of 80 can be achieved, and CISA (Cybersecurity and Infrastructure Agency) mentions DNS can reach values between 28 and 54. The traffic that the top DNS hosts can generate can be destructive; an attacker capable of generating 100 megabits/second in queries can generate traffic as high as 3.89 terabits/second at the victim, as showcased by Table~\ref{tab:data}. This puts the DNS amplification provided by the top hosts located in Greece in a class of its own compared to the impactful Memcached attack from 2018, which reached a peak traffic of 1.3 Tbps~\cite{akamai2018attackspotlight}.  


\textbf{Mitigations}. As effective campaigns performed in the past managed to mitigate the dangers of NTP~\cite{exit_hell} and Memcached~\cite{akamai2018attackspotlight}, we hope that the cybersecurity community will continue to hold similar initiatives for DNS. We propose restricting the use of ``ANY'' queries as they are bound to lead to high amplification, which has also been suggested by an RFC~\cite{rfc-8482}. Even if extensive records need to be sent, we advise following the DNS flag day 2020 recommendation of setting the EDNS0 buffer size to 1,232 and adequately configuring it to send larger responses over TCP. Finally, we want to encourage every ISP to implement BCP38~\cite{ferguson-2000} to fight IP spoofing, which would lead to the extinction of spoof-based attacks. 

\textbf{Limitations}. Our work has only explored three protocols, but many other protocols that could be exploited exist (such as CharGen or SNMP~\cite{amplification_hell}). Furthermore, our results should serve as a lower bound to what an attacker could achieve in reality, as an attacker might further optimise and achieve a higher BAF for hosts than we achieved a small one. We have also only analysed the IPv4 range, as there were no IPv6 hosts on Censys running one of the studied protocols. Still, we want to highlight the need for further studies for IPv6 hosts. Lastly, a worldwide study would have to be conducted to confirm or contradict patterns found in our work, such as the influence of the EDNS0 buffer size.  
